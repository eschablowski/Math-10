\title{Notes for Chapter 2: Basic Structures}
\author{Elias Schablowski}
\documentclass{article}

\usepackage{amsmath,amsthm,amssymb,amsfonts}
\usepackage{hyperref}

\begin{document}
    \section{2.1 - Sets}
        Sets are groups of objects, often Numbers.
        Objects in sets usually have something in common, as otherwise a set would be nonsensical.
        \subsection{Common Sets}
            \begin{enumerate}
                \item The set of all Complex numbers - $\mathbb{C}$ (\textbackslash mathbb\{C\})
                \item The set of all Real numbers - $\mathbb{R}$ (\textbackslash mathbb\{R\})
                \item The set of all Rational numbers - $\mathbb{Q}$ (\textbackslash mathbb\{Q\})
                \item The set of all Integer numbers - $\mathbb{Z}$ (\textbackslash mathbb\{Z\})
                \item The set of all Natural numbers - $\mathbb{N}$ (\textbackslash mathbb\{M\})
                \item The set of all Imaginary numbers - $\mathbb{I}$ (\textbackslash mathbb\{I\})
                \item The empty Set - $\emptyset$ or $\{\}$ (\textbackslash emptyset or \textbackslash \{ \textbackslash \})
            \end{enumerate}
            $$\mathbb{N} \in \mathbb{Z} \in \mathbb{Q} \in \mathbb{R} \in \mathbb{C}$$
            $$\mathbb{I} \in \mathbb{C}$$
        \subsection{Subsets}
            The set $A$ is a \textit{subset} of $B$ iff $B$ contains \textbf{all} objects of $A$.
            $A$ being a subset of $B$ can also be expressed as $B$ is a \textit{superset} of $A$.
            Subsets are written as $\in$ (\textbackslash in) in equations.
        \subsection{Size of a Set}
            $| S |$ denotes the size, or number of \textit{distinct}\footnote{Distinct objects means that equivalent objects are only counted once, e.g. $|{2, 2}| = 1$ as $2$ is only counted once} objects in set $S$.
            Size is also called the \textit{Cardinality} of a \textbf{finite} set.
        \subsection{Power Sets}
            A Power Set is one that includes all subsets of another set.
            Power Sets are denoted by $\mathcal{P}(S)$ (\textbackslash mathcal \{ P\} (S)) in equations.
        \subsection{Ordered n-tuple}
            An ordered n-tuple is an ordered collection denoted by $(a_1, a_2, ...)$ ((a\_1, a\_2, ...))
            Ordered tuples are equal if they have the same elements at the same locations.
\end{document}